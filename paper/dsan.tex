\documentclass[10pt,twocolumn,letterpaper]{article}

\usepackage{iccv}
\usepackage{times}
\usepackage{epsfig}
\usepackage{graphicx}
\usepackage{amsmath}
\usepackage{amssymb}
\usepackage{bbm}
%\usepackage{ccmap}
% Include other packages here, before hyperref.

\usepackage{color}
\usepackage{enumerate}


\newcommand{\cxj}[1]{\textcolor[rgb]{1.00,0.00,0.00}{(cxj: #1)}}
\newcommand{\mdf}[1]{\textcolor[rgb]{0.00,0.00,1.00}{#1}}
\newcommand{\bobo}[1]{\textcolor[rgb]{0.00,0.30,0.00}{(mbobo: #1)}}
\newcommand{\que}[1]{\textcolor[rgb]{0.00,0.30,0.00}{#1}}
\newcommand{\vb}[1]{\mathbf{#1}}
\newcommand{\comments}[1]{}


% If you comment hyperref and then uncomment it, you should delete
% egpaper.aux before re-running latex.  (Or just hit 'q' on the first latex
% run, let it finish, and you should be clear).
\usepackage[pagebackref=true,breaklinks=true,letterpaper=true,colorlinks,bookmarks=false]{hyperref}

% \iccvfinalcopy % *** Uncomment this line for the final submission

\def\iccvPaperID{****} % *** Enter the ICCV Paper ID here
\def\httilde{\mbox{\tt\raisebox{-.5ex}{\symbol{126}}}}

% Pages are numbered in submission mode, and unnumbered in camera-ready
\ificcvfinal\pagestyle{empty}\fi
\begin{document}

%%%%%%%%% TITLE
\title{Deep Shape-Aware Network for Objects Segmentation}

\author{First Author\\
Institution1\\
Institution1 address\\
{\tt\small firstauthor@i1.org}
% For a paper whose authors are all at the same institution,
% omit the following lines up until the closing ``}''.
% Additional authors and addresses can be added with ``\and'',
% just like the second author.
% To save space, use either the email address or home page, not both
\and
Second Author\\
Institution2\\d
First line of institution2 address\\
{\tt\small secondauthor@i2.org}
}

\maketitle
%\thispagestyle{empty}


%%%%%%%%% ABSTRACT
\begin{abstract}
   Accurate segmentation of biomembrane objects is a crucial step to obtain morphological statistics in biomedical analysis.
   In many scenarios, prior shape knowledge about the target objects is available, which is valuable for resolving the coarse boundary and touching problem.
   In this paper, we introduce a novel Deep Shape-Aware Network (DSAN) by incorporating prior shape knowledge about plausible components into the network.
   Our DSAN is a multi-task framework that simultaneously predicts an objectness score and several shape parameters as auxiliary at each pixel.
   The shape parameters formulate the shape of each object, which can not only separate segmented objects into individual ones, but also optimize their boundaries in terms of prior shape knowledge.
   Furthermore, a novel joint pooling operation is developed to optimize both objectness scores and shape parameters to produce better segmentation masks.
   The whole process is efficient and trained end-to-end.
   Experiments on two challenging biomedical problems, including synaptic vesicles and histopathological glands segmentation, demonstrate the effectiveness of our approach.
   And we expect our method to empower more work to utilize prior shape knowledge into neural networks to benefit their researches.

\end{abstract}

%%%%%%%%% BODY TEXT
\section{Introduction}
Recent advances in biomedical image analysis have assisted many pathologists and biologists to facilitate their researches \cite{Xu2016, Ronneberger2015,Chen2016c,Lieman-Sifry2017,Paszke2016}.
Among these researches, a significant application is to obtain the accurate segmentation of specific membrane structure in a biomedical image, such as lumenal glands, synaptic vesicles and cells.
Especially, the morphological shape and spatial distribution of synaptic vesicles are helpful to study the neural activity in different brain regions, while \mdf{morphological statistics of lumenal glands are widely used for the assesment of the malignancy degree of adenocarcinomas.} 
%
Conventionally, these crucial steps are performed by human expert, which are time-consuming and suffer from subjective factors.
\cxj{For example, how long does it take to handle a task..?}
\mdf{Therefore, it is significantly demanded to improve the efficiency as well as the reliability with automatic segmentation methods. }
%[copy]Therefore automatic segmentation methods are highly demanded to improve the efficiency as well as reliability and reduce the workload on experts[copy].

\begin{figure}
\begin{center}
\includegraphics[width=3.3in]{figures/Fig1.pdf}
\end{center}
   \caption{Examples illustrating deficiency of existing methods for biomedical segmentation. First two rows are dense synaptic vesicle with regular shape, and bottom two rows are benign and malignant gland. (a) biomedical image; (b) results from DCAN \cxj{cite the reference}; (c) results from SCNN by incorporating prior shape constraint; (d) annotations by experts. \cxj{what are the blank areas for the last two rows?}}
   \label{FigImgs}
\end{figure}

\mdf{However, it is non-trivial to automatically segment biomedical images.} 
%[copy]However, there exists several challenges in these tasks.[copy]
First, biomedical images are usually noisy and of low contrast, because of deficient imaging techniques, as shown in Figure~\ref{FigImgs} (a).
%
Second, due to the compact and dense arrangement of majority membrane structures, it is hard to separate objects individually, which is known as the touching problem.
%
Third, as most deep neural architecture of biomedical images are based on fully convolutional networks~\cite{Long2015}, they inevitably suffer from poor localized object boundaries caused by large receptive fields and many pooling layers.
\cxj{put this sentence to the next paragraph?}



Recently, deep neural networks have demonstrated excellent performance in biomedical image segmentation with the use of fully convolutional networks \cite{Long2015,Dhungel2015,Ronneberger2015,Roth2015,Chen2015,Lieman-Sifry2017,Xu2016}.
\mdf{However, the pooling and downsampling layers in these networks make it inevitable to get poorly localized object boundaries.}
%
\mdf{To increase the boundary accuracy, many efforts have been made recently. A U-shaped deep network called U-net~\cite{Ronneberger2015} is proposed for biomedical image segmentation. By employing the skip connections between the contracting and expanding paths, context information can be directly propagated to higher resolution layers for detail preserving.
%
Several improvements of U-net were proposed soon afterwards. 
DeepVentricle~\cite{Lieman-Sifry2017}, which uses the same padding instead of valid padding, has been successfully used for cardiac segmentation.
Recently, DCAN~\cite{Chen2016a} integrates the complementary information of object regions and contours in a multi-task learning framework. \cxj{for more accurate boundaries?}
}
%
Although these methods achieved promising results in their segmentation tasks, they may fail to achieving satisfying performance in denser, smaller objects with regular shapes, as Figure~\ref{FigImgs} shows.
%
\mdf{More specifically}, segmenting synaptic vesicles in our task raises higher demand on localizing contour for each vesicles.



%
\comments{
For this reason, [copy]\cite{Ronneberger2015} proposed the U-net that designed a U-shaped deep convolutional network for biomedical image segmentation.[copy]
[copy]It uses skip connections between the contracting and expanding paths to directly propagate context information to higher resolution layers to preserve details.[copy]
Later, a UNet variant, DeepVentricle \cite{Lieman-Sifry2017}, has been used for cardiac segmentation, which used same padding instead of valid padding.
Further improvements have been shown in DCAN \cite{Chen2016a}, which [copy]investigates the complementary information of objects and contours under a multi-task learning framework.[copy]
Specially, [weak copy]DCAN simultaneously segment the object and separate the clustered objects into individual ones with the help of their contours[weak copy].}
%



\begin{figure*}\label{FigSCNN}
\begin{center}
\includegraphics[width=6.8in]{figures/Fig2.pdf}
\end{center}
   \caption{Overview of our proposed scnet. Given an image (a), multi-task neural network simultaneously predict a coarse segmentation (d) and parameterized contour description (d) using shared feature maps (b).
   Then a joint max pooling is applied to pool (c) with (d) and output new parameterized contour description (e).
   Finally, segmentation (f) is obtained by optimizing coarse segmentation (d) with the parameterized contour description (e).}
\end{figure*}

In this paper, we propose the first \mdf{Shape-Constrained} neural network (SCNN) to segment dense objects by inherently incorporating prior shape knowledge into the network.
%while still accommodate serious shape deformation.
Similar with \cite{Chen2016a}, we formulate the network as a multi-task learning framework by simultaneously predicting a segmentation map and an auxiliary map.
%The auxiliary map, usually being contours map \cite{Chen2016a}, \cite{Chen2016}, \cite{Bertasius2016} ,is used to supplement complementary information for segmentation map.
Instead of predicting the contour probability map, as used in \cite{Chen2016a,Chen2016,Bertasius2016}, our SCNN learn\mdf{s} a parameterized description of the contour shape as \mdf{a} auxiliary map, which emphasizes more on \mdf{the} overall shape of objects.
%^In our framework, instead of directly predicting object contours, our SCNN learn the parameterized expression of contours, which emphasizes more on the overall shape of object.
%Especially, our parameterized contour expression can not only separate objects into individual ones, but also be embed in different shape constraint by defining different parameter composition.
The complementary information in parameterized contour description can not only separate objects into individual ones, but also optimizes the contours shape.
%
However since there existing some seriously deformable objects, contours shape \mdf{cannot} be parameterized uniformly and accurately.
To this end, we select a best representative shape as constraints and only modify segmentation predictions in ambiguous regions, where usually contains contours \cxj{what do you mean here?}, using the predicted parameterized contour description.
%In this way, most biomedical objects with regular shape, such as ellipse, are beneficial, while the segmentation of fractional deformable object are not obviously affected.

However, predicting parameterized contour description over the whole map is \mdf{significantly more challenging than generating} contour probability map.
\cxj{Why this is more difficult? what is the challenge?}
%Eespecially predictions within one object region it is hard to guarantee the predictions belonging to one object corresponding to an identical shape, and performance will drop obviously in contours area, which is the exact the area where we desire to modify the segmentation result.
Therefore, we proposed a novel joint max pooling (JMP) layer to only predict the contour description in center region of objects and fill the rest region with them.
Furthermore, JMP is designed as a trainable layer, of which the back propagation is benefit to both segmentation and parameterized contours description.
%Furthermore, it is designed as a trainable layer, which can be extended to any networks.
%Explicitly, the objectness map can be used to substitute the contours expressions belonging to one object region with the expression of the highest objectness score, which reduces the learning difficulties and improve the inference accuracy.
%And contours expression map will increase the penalties on objectness loss of the regions, where there exist disagreements between objectness prediction and parameterized countour prediction in trun.
%Our JMP can be implemented as a layer, which can be trained end-to-end and adopted to other multi-task networkds.

Overall, the contribution of this paper is three-fold:
\begin{enumerate}
	\item We first effectively incorporate shape constraint into deep neural networks.
	% for biomedical image segmentation.
	\item We propose a novel joint max pooling for benefiting both multi-task outputs.
	\item Our framework is applicable to a series of different tasks such as biomedical image segmentation, scent detection task and achieves the state-of-the-art performance.
\end{enumerate}


\comments{
3) achieving better performance on diverse biomedical segmentation tasks,
4) in experiments, we show that our method can be extended to scent detection task, which obtains the state of art performance
}

\section{Related Work}
\textbf{Biomedical image segmentation} Convolutional neural networks have shown great improvement in many segmentation tasks in medical image analysis \cite{Dhungel2015a,Roth2015a,Roth2016,Chen2016e,Nogues2016,Dou2016,Qin2016,Chen2017,Ronneberger2015,Lieman-Sifry2017,Chen2016c}.
However, because of the pooling and downsampling layers used in FCN \cite{Long2015}, the localized object boundaries are usually poor and coarse, which aggravate the touching problem \cite{Dou2016,Chen2017,Ronneberger2015,Lieman-Sifry2017,Chen2016c}.
To this end, the U-net was proposed by employing a U-shape deep convolutional network for biomedical image segmentation and obtained state-of-the-art performance on several challenges.
Although the raw context information can be directly propagated to higher resolution layers for detail preserving by U-net, the utilization of weighting losses on boundaries may not satisfactorily handle the touching problem.
Soon several improvement on U-net were proposed \cite{Lieman-Sifry2017,Chen2016c,Cicek2016}.
\cite{Chen2016c} propose the kU-net by employ multiple submodule U-nets to work on different image scales, \cite{Cicek2016} extented the U-net to 3D applications and \cite{Lieman-Sifry2017} substitute same padding with valid padding in cardiac segmentation.
Recently, DCAN \cite{Chen2017} proposed a multi-task learning framework to solve the touching problem by incorporating the complementary objects contour into the model, which inspired us, and their method obtained state-of-the-art performance in the challenging gland segmentation task.

\textbf{Shape constraints for segmentation} Instead of directly learning from the raw images, many methods have been attempted to introduce the prior knowledge of plausible objects.
\cite{Delgado-Gonzalo2012} was designed for segmenting elliptical objects sequentially and \cite{Veksler2008a} introduced the star convexity priors into their model by restricting all rays emanating from a user-defined central point.
Soon, the approach in \cite{Veksler2008a} was extended to Geodesic Star Convexity constraint by \cite{Gulshan2010a}, and \cite{Gorelick2014a} succeeded in introducing the convex shape constraints into segmentation task.
Recently, \cite{Royer2016a} is able to handle many generic convex objects of multiple foreground classes automatically.
Although, the existing methods gave the satisfying performance in their task, they did use the powerful feature representation capability of CNN model.
Thus in this paper, we first introduce the shape constraint into the neural network for biomedical segmentation, of which the prior shape knowledge about plausible objects are usually available.

\section{Proposed Method}
\label{sec:method}



A complete pipeline of Shape-Constrained Neural Network (SCNN) is illustrated in Figure \ref{FigSCNN}.
The framework is trained end-to-end and consists of three key components:
1) a multi-task neural network based on FCN,
2) proposed joint max pooling and
3) optimizing segmentation result with parameterized contour description.
%
\mdf{Given an input image, the first part is multi-task FCN which separately predicts a binary segmentation and shape parameter maps (Sec.~\ref{sec:multi-task-fcn}).  The two branches integrate with each other by a joint max pooling layer and contour optimization step to finally predict the shape-constrained sementation (Sec.~).}
%First, we start by introducing our multi-task neural network in Sec. \ref{Sec21}.
%And then, joint max pooling is described in detail as a tainable layer in Sec. \ref{Sec22}.
%Finally, we show how to utilize the complementary information of parameterized contour expressions to improve segmentation performance in Sec. \ref{Sec23}.



\subsection{Multi-task FCN}
\label{sec:multi-task-fcn}

\begin{figure}\label{FigMTN}
	\begin{center}
		\includegraphics[width=3.3in]{figures/FigMTN.pdf}
	\end{center}
	\caption{Our multi-task network architecture. \cxj{I would sugest using real images as input and output.}}
\end{figure}

%Due to the essential ambiguity in touching regions caused by segmentation map, complementary information is needed to separate clustered objects into individuals ones.

The architecture of our multi-task learning network is shown in Figure \ref{FigMTN}.
It simultaneously predicts a segmentation map $\mathbf{M}_{seg}$ and an auxiliary map $\mathbf{M}_{aux}$ to provide complementary information.
% 
In this network, the feature learning part is shared and based on the publicly available DeepLab model~\cite{Chen2014a}, which introduces zeros into the filters to enlarge its field-Of-View \cxj{perceptive field? Why zeros in the filter can enlarge FOV?}.
%
It is initialized from VGG-16 ImageNet pretrained model.
Subsequently, feature maps output by last shared conv layer are fed into two individual branches.
In each branch, successive two conv layers, respectively with kernel size of $3\times3$ and $1\times1$, are applied to input map, and then an upsampling layer restores their resolution to the input image size.
%Especially, one branch predicts $\mathbf{O}^1_{seg}$ as a dense classification task, and the other branch predicts $\mathbf{O}^1_{aux}$ as a regression task.
During training, the parameters of shared network are jointly optimized, [copy] while the parameters of two individual branches are updated independently.

Instead of directly predicting contour probabilities like \cite{Chen2016a,Xu2016}, we choose parameterized contour description as our complementary information, which emphasizes more on the overall shape.
%
Specifically, an ellipse shape is depicted by five parameters: $\Theta = \{\theta, x_c, y_c, a, b\}$, where $\theta$ is the rotated angle of the major axis \cxj{from x-axis?}, $x_c$ and $y_c$ are the coordinates of its center, and $a$, $b$ are respectively \mdf{the lengths of the} major and minor axis.
%
Different definitions of parameters means different shape prior knowledge of object contour.
Similar to \cite{Redmon2016}, the predicted contour description on $(x,y)$ is expressed by $\mathbf{O}_{aux}(x,y) = [\theta, dx_c, dy_c, a, b]$, 
\cxj{why only on $x,y$?} 
where
\begin{eqnarray}\label{EqMax}
\begin{aligned}
\theta &= \theta^*\\
dx_c &= (x-x_c^*)/width\\
dy_c &= (y-y_c^*)/height\\
a &= a^*/width\\
b &= b^*/height\\
\end{aligned}
\end{eqnarray}
[$\theta^*$, $x_c^*$, $y_c^*$, $a^*$, $b^*$] are parameters depicting the true ellipse shape of the object to be segmented.
$width$ and $height$ are image size.
\cxj{I am confused here...}


The objective function follows the multi-task loss in Faster R-CNN~\cite{Ren2015}.
Our loss function for an image is defined as:
\begin{eqnarray}\label{EqLoss}
\begin{aligned}
L(\mathbf{O}_{seg},\mathbf{O}_{aux}) &= L_{seg}(\mathbf{O}_{seg},\mathbf{O}_{seg}^*)+\\
&\lambda L_{aux}(\mathbf{O}_{aux},\mathbf{O}_{aux}^*£¬\mathbf{O}_{seg})
\end{aligned}
\end{eqnarray}
where $L_{aux}(\mathbf{O}_{aux},\mathbf{O}_{aux}^*£¬\mathbf{O}_{seg}^*)$ only compute loss on regions with positive $\mathbf{O}_{seg}^*$ and $\lambda$ is a balancing weight.




\subsection{Joint Max Pooling}
\label{sec:joint-max-pooling}

\mdf{Based on the binary map of segmentaion and auxilary map of shape parameters obtained from the two individual branches, we now integrate these two types of information to produce more accurate boundaries.}
%
We introduce a novel joint max pooling to improve both the accuracy of segmentation and parameterized contour predictions.
Different from conventional max pooling layers, our JMP takes two inputs and pooling one with the other one.
\cxj{not clear about your key idea. You need one sentence to describe your main idea of joint pooling, how does it combine two maps..}
%JMP takes in the two outputs from previous multi-task neural networks and outputs two new predictions, as shown in Figure \ref{Fig_scnet}.

A conventional pooling operation can expressed as 
\begin{eqnarray}\label{pooling}
\begin{aligned}
y_{i,j} = \sum_{p\in \mathcal{N}_{i,j}} \omega_{p}x_{p},
\end{aligned}
\end{eqnarray}
where $\mathcal{N}_{i,j}$ a neighbor region of pixel $(i,j)$ according to the sliding window, and $\omega_{p}$ is the weight of pixel $p$.
%
For traditional max pooling, $\omega_p \in {0,1}$ is a binary indicator for that if $x_p$ is the maximum in the local region.
There is only one pixel in the neighborhood has $\omega_p=1$ and all the others have $\omega=0$.
%
For an average pooling, all pixels in the local window take the identical weight $\omega_p=\frac{1}{N}$, where $N$ is the total number of pixels in the local region $\mathcal{N}$.
%
Intuitively, $\omega$ acts like an "indictor" determining which information in $\mathbf{\overline{X}}$ should be propagated to the next layer.
Based on this observation, we proposed a split version of pooling by obtaining $\mathbf{S}$ from an independent input, instead of $\mathbf{\overline{X}}$.
%This novel split.
\begin{figure}\label{FigJMP}
\begin{center}
\includegraphics[width=3.4in]{figures/Fig3.pdf}
   %\includegraphics[width=0.8\linewidth]{egfigure.eps}
\end{center}
   \caption{An example of joint max pooling.
   Two windows of same size synchronously slide on $\mathbf{X}$ and $\mathbf{S}$.
   Top window will propagate the element, of which the position corresponding to bottom window has a maximum value, to next layer.
   }
\label{F3}
\end{figure}



In practice, two windows with the same size synchronously slide on two independent input.
One of the window is denoted as indicating window acting as the "indictor", while the other one is denoted as pooling windows, of which the useful information should be propagated to next layer.
In other word, elements in the indicating window determine the pooling strategy in pooling windows.
A simple example is illustrated in Figure~\ref{FigJMP}.

However for max pooling, $\mathbf{S}$ is hard to be directly learned to be binary.
Therefore we add a threshold function by:
\begin{eqnarray}\label{jmp}
\begin{aligned}
y_{\mu,\nu} &= \sum_{i,j}x_{i,j}G(s_{i,j})~~~~x_{i,j}\in \mathbf{\overline{X}},s_{i,j}\in \mathbf{S} \\
G_{\mathbf{S}}(s_{i,j},)&=\left\{\begin{array}{cc}
1&if~s_{i,j}>=max(\mathbf{S})\\
0&else\\
\end{array}\right.
\end{aligned}
\end{eqnarray}

From Figure \ref{FigJMP}, it should be noted that most elements in $\mathbf{X}$ have been substituted with the element of which the position corresponding to maximum in $\mathbf{S}$.
Therefore in our SCNN, if $\mathbf{O}_{seg}$ is denoted as $\mathbf{S}$ and $\mathbf{O}_{aux}$ is denoted as $\mathbf{X}$, only the contour description in $\mathbf{O}_{aux}$ with a local maximum objectness in $\mathbf{O}_{seg}$ can be retained by JMP.
Moreover, the discarded elements in $\mathbf{O}_{aux}$ will be replaced by a nearest retained contour description when pooling stride is set to be $1$.
%
And positions with local maximum objectness usually correspond to the center region of objects.
Especially, the kernel size and iterations of JMP determine how far a retained contour expression can be spread.
JMP guarantees the accuracy and consistency of contour description in ambiguous region of an object.

One important contribution of our JMP is that the residual error can be correctly back propagated to its inputs.
This makes it a trainable layer in any network architecture and our SCNN become a fully trainable system.
Defining $L_\mathbf{X}$ as the residual error on $\mathbf{X}$ , the back propagation for $x_{i,j}$ can be expressed by:
\begin{eqnarray}\label{bpx}
\begin{aligned}
\frac{\partial L_\mathbf{X}}{\partial x_{i,j}}=\frac{1}{m}\sum\limits_{y_{\mu,\nu}\in\mathbf{U}}\frac{\partial L_\mathbf{X}}{\partial y_{\mu,\nu}}G_{\mathbf{S}_{u,v}}(s_{i,j})\\
\end{aligned}
\end{eqnarray}
where $\mathbf{U}$ is output set $\{y_{\mu.\nu}\}$ associated with $x_{i,j}$ and $m$ is the size of $\mathbf{U}$.
${S}_{u,v}$ is the corresponding pooling window centered on ${u,v}$ of $\mathbf{X}$.
Different from conventional max pooling, Eq.~\ref{bpx} converge the gradients on the positions with local maximum $s_{i,j}$£¬ which usually are the centers of object.
%
Implementing Eq. \ref{bpx} to our SCNN can make it only focus on predicting accurate parameterized contour description on the center area of objects, instead of the whole region.


%Specially in Figure \ref{FigSCNN}, the input segmentation map is assumed to not only influence the output but also feeds a subsequent layers, thus also receiving gradient contributions $\frac{\partial L}{\partial s_{i,j}}$ from the next layer during back-propagation.

Defining $L_\mathbf{S}$ as the residual error on $\mathbf{S}$.
We assume that $s_{i,j}$ not only influences the following $y_{i,j}$ but also feeds a subsequent layer in Figure \ref{FigSCNN}, thus also receiving gradient contributions $\frac{\partial L_\mathbf{S}}{\partial s_{i,j}}$ from that layer during back-propagation.
The back propagation for $s_{i,j}$ is formulated by
%
\begin{eqnarray}\label{bps}
\begin{aligned}
\frac{\partial L_\mathbf{S}}{\partial s_{i,j}}&=\frac{\partial L_\mathbf{S}}{\partial s_{i,j}}+\frac{1}{m}\sum_{y_{\mu,\nu}\in\mathbf{U}}\frac{\partial L_\mathbf{X}}{\partial y_{\mu,\nu}}\frac{\partial y_{\mu,\nu}}{\partial s_{i,j}}\\
&=\frac{\partial L_\mathbf{S}}{\partial s_{i,j}}+\frac{1}{m}\sum_{y_{\mu,\nu}\in\mathbf{U}}\frac{\partial L_\mathbf{X}}{\partial y_{\mu,\nu}}x_{i,j}\frac{\partial G_{\mathbf{S}_{u,v}}(s_{i,j})}{\partial s_{i,j}}\\
&=\frac{\partial L_\mathbf{S}}{\partial s_{i,j}}+\sum_{y_{\mu,\nu}\in\mathbf{U}}\frac{1}{m}\frac{\partial L_\mathbf{X}}{\partial y_{\mu,\nu}}x_{i,j}\delta_{\mathbf{S}_{u,v}}(s_{i,j})\\
\end{aligned}
\end{eqnarray}
where $\delta_{\mathbf{S}_{u,v}}(s_{i,j})$ is the derived function of $G_\mathbf{S}(s_{i,j})$, which has an infinite response when $s_{i,j}=max(\mathbf{S}_{u,v})$.
In order to normally back propagate, $\frac{\partial L_\mathbf{S}}{\partial s_{i,j}}$ is approximated by:
\begin{eqnarray}\label{dG}
\begin{aligned}
\frac{\partial L_\mathbf{S}}{\partial s_{i,j}}&= \frac{\partial L_\mathbf{S}}{\partial s_{i,j}}(1+\frac{1}{m}\sum_{y_{\mu,\nu}\in\mathbf{U}}\lambda\widetilde{\delta}_{\mathbf{S}_{u,v}}(s_{i,j}))\\
\widetilde{\delta}_{\mathbf{S}_{u,v}}(s_{i,j})&=\left\{\begin{array}{cc}
1&if~s_{i,j}=max(\mathbf{S})\\
0&else\\
\end{array}\right.
\end{aligned}
\end{eqnarray}

Intuitively, Eq. \ref{dG} add\mdf{s} a loss weight on gradients of \mdf{the} local maximum $s_{i,j}$ with the control $\lambda$ (set according to iterations of JMP) to avoid false detection as much as possible.



\subsection{Fusion for Final Segmentation}
With the predicted probability maps of objectness $\mathbf{O}_{seg}$ and parameterized contour description $\mathbf{O}_{aux}$ from SCNN, the final segmentation $m(i,j)$ can be obtained by fusing them together:
%
\begin{eqnarray}\label{fusion}
\begin{aligned}
m(i,j)=\left\{\begin{array}{cc}
1&if~\mathbf{O}_{seg}(i,j)>\tau_2\\
0&if~\mathbf{O}_{seg}(i,j)<\tau_1\\
f(\mathbf{O}_{aux}(i,j))&else\\
\end{array}\right.
\end{aligned}
\end{eqnarray}
%
where $\tau_2$ and $\tau_1$ are two thresholds (set empirically) to control the degree of object contour modification by $\mathbf{O}_{aux}(i,j)$.
%
$f(\mathbf{O}_{aux}(i,j))$ is a function judging whether a position is within a shape by its coordinate $(i,j)$ and shape description $\mathbf{O}_{aux}(i,j)$.
For example in our task, we define an ellipse by $\mathbf{O}_{aux}(i,j) = [\theta, x_c, y_c, a, b]$, therefore the function is expressed by:
\begin{eqnarray}\label{fusion}
\begin{aligned}
f(\mathbf{O}_{aux}(i,j))&=\left\{\begin{array}{cc}
1&if~\frac{dx^2}{a^2}+\frac{dy^2}{b^2}<1\\
0&else\\
\end{array}\right.\\
dx &= cos(\theta)(i-x_c)+sin(\theta)(j-y_c)\\
dy &= -sin(\theta)(i-x_c)+cos(\theta)(j-y_c)\\
\end{aligned}
\end{eqnarray}

Our fusion strategy can appropriately utilize prior shape knowledge to optimize segmented object.
It can not only separate objects into individual ones, but also optimize most regular object without losing generalization to deformable objects.
The SCNN can be easily extended to other shape constraint, \mdf{such as rectangle shape, circles}, if the shape can be parameterized.

\section{Experiments}
\label{sec:results}
To demonstrate our superiority over existing methods on segmentation, we first present results on two diverse biomedical image segmentation problems, including synaptic vesicle segmentation and gland segmentation.
To demonstrate the easy extension and generic applicability of our framework, we modified our SCNN to scene text segmentation task.

\subsection{Synaptic vesicle segmentation}
\noindent\textbf{Dataset}
Synaptic vesicle is a good example to evaluate our method, as most of vesicle shape are ellipse.
The images were acquired by . \cxj{by what?}
However synaptic vesicle images are much noisy due to.
The plausible vesicles in image are densely arranged and easily confused with other membrane in presynaptic cell, therefore it is hard to obtain clear contour for such dense and small objects.
%The ground truths of dataset are held out by biologists for objective evaluation.
The original dataset is composed of $120$ images with annotations provided by biologists.
The height and width of each image are respectively $1019$ and $1053$, averagely containing $200$ vesicle objects.
Since the average length of a vesicle is about $30$ pixel, we crop a $321\times 321$ region from the original image as our input to network so that each cropped image contain about $25$ objects.
We uniformly crop $25$ patches in each original image with overlap, then the final dataset consists of $3000$ images of $321\times 321$ resolution.
\cxj{image size?}

Similar with many existing approaches, we utilize a data augmentation process to reduce the overfitting and increase the robustness of our network.
In the data augmentation, translation, rotation and image flipping are used, which generate total $8787$ images.
\cxj{The final dataset size?}

For better evaluation, a six-fold cross validation is applied in our experiments.
The first five out of six images are prepared to train our model and the rest of them are used to test its performance.
\cxj{Six-fold cross validation? or other strategy? }
%
The validation processing has been repeated several times and the average performance will be reports.


\noindent\textbf{Implementation details.}
We train our network on the open-source deep learning library Caffe~\cite{Jia2014}.
%All the experiments are implemented on a workstation with TitanX GPU cards.
The model of our SCNN is well trained by two-step process.
In first stage, we independently train the segmentation branch and shared layers as a general segmentation task.
The parameters of shared layers are initialized from VGG-16 ImageNet pretrained model.
During training phase, the base learning rate was set as $10^{-3}$ and a 'step' policy is adopted by decreasing the learning rate by a fact of $10$ every $2000$ iterations.
And mini-batch size was set to $30$ for one iteration with the max iteration number $20k$.
In the second stage, the multi-task FCN as well as two successive joint max pooling layers is jointly fine-tuned on the model from first stage.
The learning rate of new added layers in auxiliary branch was set as $10^{-5}$, while the other learning rate was set to $10^{-8}$.
The iteration number of second stage isset as $8k$.
For max pooling layers, pooling size was set as $7\times 7$ with stride $1$.
And the balancing weight $\lambda$ in Eq.~\ref{EqLoss} was set as $5$.
Finally because the shape of most vesicle contours are regular ellipse, we can impose a relative strong constraint to the shape of segmented objects.
Therefore in testing phase, the thresholds $\tau_1$ and $\tau_2$ in local fusion step are respectively set to $0.2$ and $0.9$.

\noindent\textbf{Evaluation setup.}
%
The evaluation criteria in our experiments includes an score $F_1$ to evaluate performance of object detection and a pixel intersection-overunion (IOU) averaged across different classes to evaluate the segmentation accuracy .
%The $F_1$ score considers the performance of object detection, while IOU consider the segmentation performance, respectively.

For detection, the $F_1$ score is defined as:
\cxj{Can we use a more intuitive symbol, say $R_{detection}$?}
%
\begin{eqnarray}\label{fusion}
\begin{aligned}
F1 = \frac{2PR}{P+R}
\end{aligned}
\end{eqnarray}
where $P$ is the detection precision and $R$ is the detection recall.
Especially, a true positive detection is the segmented object that intersects with at least $50\%$ with the ground truth, otherwise it is regarded as a false positive.
If a ground truth object has no corresponding segmented object that intersects more than $50\%$, it is regarded as a false negative.

The IOU metric for segmentation accuracy is defined as:
%
\begin{eqnarray}\label{fusion}
\begin{aligned}
IOU = \frac{1}{N_s}\sum_{i=1}^{N_s}\frac{G_i\bigcap S_i}{G_i\bigcup S_i}
\end{aligned}
\end{eqnarray}
%
where $N_s$ denotes the number of label classes.
$G_i$ denotes the pixel set in ground truth of $i$-th label.
$S_i$ denotes the pixel set in segmented map of $i$-th class.
In our object segmentation task, there are two kind of labels: object or background, therefore $N_s$ is set to $2$.
\cxj{What do you mean by "classes" here? instances? }

\begin{figure*}
    \begin{center}
        \includegraphics[width=6.8in]{figures/FigVesicle.pdf}
    \end{center}
    \caption{Some segmentation results of synaptic vesicle: the image from top to bottom are respectively input image, segmentation of u-net, DCAN, SCNN and ground truth.}
    \label{FigVesicle}
\end{figure*}


\noindent\textbf{Qualitative evaluation on synaptic vesicle segmentation}
Figure \ref{FigVesicle} shows some segmentation results of testing data.
In order to verify the effectiveness of parameterized shape information, we compared our method with U-net \cite{Ronneberger2015} without any complementary information and DCAN~\cite{Chen2016a} that uses auxiliary contour map.

From segmentation results we can see that without other complementary information, there exists many touching objects that cannot be separated by U-net (second row in \ref{FigVesicle}).
This is because that the contours of many synaptic vesicles are very fuzzy and even not complete that miss a part of membrane due to deficient imaging technology shown in first row in \ref{FigVesicle}.
Furthermore, as synaptic vesicles usually gather around presynaptic membrane, they are crowded together which increase the difficulty to separate them into individual ones. 
Although DCAN is capable to separate touching synaptic vesicles apart in the third row of Figure \ref{FigVesicle}, it produced additional false positive detections between two touching vesicles.
The reason is that connection part between two touching vesicles are wide and long caused by terrible image quality, which can not be cleaned out by predicted contours.
Therefore in \cite{Chen2016a}, that the touching glands are closed is also a factor of success of DCAN, and we will present our SCNN still works for segmenting closed touching objects in our second experiments.  
Differen from above two methods, our SCNN uses the shape parameters of objects as complementary information to separate the touching object clearly without any sequela.
The information in shape parameters of an object are more abundant, including the prior shape knowledge of segmented objects and main dominion belonging to the object.
As the results shown in forth row of Figure~\ref{FigVesicle}, all the touching objects have been well separated and their shape are more smooth and regular.
This demonstrates the superiority of our SCNN in segmenting densely arranged objects with regular and small objects by incorporating the prior shape knowledge into network.

\noindent\textbf{Quantitative evaluation.}
To quantitatively evaluate our method, we compare the object detection rate $F_1$ and the segmentation accuracy IOU of our SCNN with the state of the art segmentation methods based on Deeplab~\cite{Chen2014a}, u-net~\cite{Ronneberger2015} and DCAN~\cite{Chen2016b}, which are commonly used in biomedical image process.
%
Their results on our synaptic vesicle dataset are shown in Table~\ref{tab:vesicle}.
%
And we further implement two version of SCNN.
The first SCNNv1 is the implementation of multi-task neural network without joint max pooling, and SCNNv2 is the complete form.
Their results are also presented in Table~\ref{tab:vesicle} to prove effectiveness of our joint max pooling.

From F1 score obtained in Table~\ref{tab:vesicle}, the performance of SCNN surpassed the other methods.
Especially, we observed that the performance of u-net obtained about $3\%$ improvement over deeplab, both of which didn't use any auxiliary supervised information.
This arises from the fact that U-shaped deep network applied by u-net to alleviate the touching problem by supplementing many context information to higher resolutions layers.
It confirmed that deficiency of context information due to successive pooling and downsampling layers is a significant factor of touching problem in biomedical segmentation.
Furthermore, we noticed that F1 score of DCAN turns out to be the worst among all the methods.
This mainly comes from the extra patch between two objects that are original touched and separated by predicted contours, shown in Figure~\ref{FigVesicle}.
Many these extra patches increase the false positive counts of DCAN.
And it can be seen that our joint max pooling operation significantly improve the performance of using shape parameter as auxiliary information.

From IOU metrics in Table~\ref{tab:vesicle}, our SCNN again gives the best performance among various methods.
It should be noted that although the other methods also obtain a good IOU score, their contour are more coarse and irregular, which increase the difficulty of post-processing such as reconstruction of 3D synaptic vesicle structure.
And our segmented vesicle are more clear and regular. 


\begin{table}
\begin{center}
\begin{tabular}{lcc}
\hline
Method & F1 & IOU \\
\hline
DeepLab & 0.8404 & 0.8495 \\
U-net & 0.1377 & 0.8180 \\
DCAN & 0.7007 & 0.8533 \\
SCNNv1 & 0.8097 & 0.8596 \\
SCNNv2 & 0.8797 & $\mathbf{0.8642}$\\
\hline
\end{tabular}
\end{center}
\caption{The detection and segmentation results of different methods in our synaptic vesicle segmentation dataset.}
\label{tab:vesicle}
\end{table}





\subsection{Gland segmentation}
\textbf{Dataset}
In this section we present SCNN for segmenting benign and malignant gland.
We consider the public dataset of \emph{Gland Segmentation Challenge Contest} in MICCAI2015~\cite{Sirinukunwattana2015a}.
%
The training dataset is composed of 85 images, consisting of 37 benign and 48 malignant, with ground truth annotations provided by expert pathologists.
Especially, there is a huge variation of glandular morphology in malignant case, which can prove the generalization of our SCNN to irregular objects.
The same data augmentation in vesicle segmentation is implemented for a better performance.

\textbf{Implementation details}
Because there exists many irregular objects in gland images that we desire to remain more contour information obtained by object prediction, the contour modification by parameterized contour information should be relatively weaker than segmenting vesicles.
By experimental verification, we find that $\tau_2=0.7$ and $\tau_1=0.4 $ produce a better results.
\cxj{So show comparison of results using different parameters.}
%
And we still use the standard elliptic parameter as the prior shape constraint for gland, as most benign glands and few malignant glands' shape are approximate ellipses.
The other implementation settings and evaluation metrics follow the vesicle segmentation.

\textbf{Qualitative evaluation on gland segmentation}
Follow previous qualitative evaluation, we presented the results of u-net and the state of art method DCAN in gland segmentation with our SCNN in Figure~\ref{fig:com-gland}.
The first two columns are the examples of benign gland images, and the rest two columns are the examples of malignant images.
From the results, we can observed that both SCAN and SCNN can well solve the touching problem in benign and malignant cases.
However for benign case, the contours of glands obtained by SCNN are more smooth than that of DCAN.
And for malignant case, since SCNN only modify the segmentation predictions on the object border, there is no obvious deterioration compared to DCAN.

\begin{figure}
	\centering
	\vspace{0.4in}
	\caption{\cxj{Add comparison of gland segmentation with U-net and DCAN.}}
	\label{fig:com-gland}
\end{figure}

\textbf{Quantitative evaluation}
Table \ref{} shows the F1 score and IOU metric over the $Gland$ $Segmentation$ $Challenge$ $Contest$ by several commonly used biomedical segmentation methods.

\begin{table}
	\centering
	\caption{Performance comparison on gland segmentation.}
	\begin{tabular}{c|cc}
		\hline
		Method & F1 & IOU \\
		\hline
		DeepLab & 0 & 0 \\
		U-Net & 0 & 0 \\
		DCAN & 0 & 0 \\
		SCNN-v1 & 0 & 0 \\
		SCNN-v2 & 0 & 0 \\
		\hline
	\end{tabular}
\end{table}


\subsection{Scene text detection}
We further extended SCNN to scene text detection task, which


%\section{Discussion}

\section{Conclusion}
In this paper, a novel and general method is proposed for incorporating the prior shape knowledge into neural network to segment the plausible objects in biomedical images.
Based on the popular multi-task FCN in segmentation, we replace the auxiliary contour probability with the shape of objects, formulated by a set of meaningful parameters.
In order to improve the accuracy of predicted shape parameters, a novel joint max pooling layer is proposed to replace the predictions in boundary regions with those in center of an object.
Finally with our local fusion strategy, the degree of shape constraint can be flexibly adjusted in terms of different tasks.
We experiment our method on two biomedical tasks for segmenting synaptical vesicles from electron microscope images and the glandular structures from colorectal cancer tissues.
%To further prove the generality of our model to diverse shape constraint, we extend our method to the scene text detection task by segmenting the scene text region with regular shape.
All the experiments have demonstrated the effectiveness of our method on the object segmentation problem with prior shape knowledge.


%The experiments have shown great superiority of our DSAN in segmentation tasks with available prior shape knowledge.
However, our method still has some limitations and can be improved in several directions. 
First, only parameterized shape constraints are supported in our network, such as ellipse, rectangle or starriness.
%
Extending our method to more complicated shapes is a future direction.
%
Second, our method only supports one type of shape constraints into the network each time so far.
%
%If the images contain several kinds of object with different prior shape knowledge, several DSANs are needed.
In the future, we will improve our approach to support more than one type of shapes in one network for a wider range of applications.



{\small
\bibliographystyle{ieee}
\bibliography{egbib}
}

\end{document}
