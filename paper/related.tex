\section{Related Work}
%
\mdf{
Biomedical image segmentation is a traditional topic in computer vision. 
In this section, we briefly review the development of segmentation techniques and discuss related neural networks on generic image segmentation problems.}

\noindent \textbf{Biomedical image segmentation} 
\cxj{Add some traditional methods?}
%
Convolutional neural networks have shown great improvement in many segmentation tasks in medical image analysis \cite{Dhungel2015a,Roth2015a,Roth2016,Chen2016e,Nogues2016,Dou2016,Qin2016,Chen2017,Ronneberger2015,Lieman-Sifry2017,Chen2016c}.
However, because of the pooling and downsampling layers used in FCN \cite{Long2015}, the localized object boundaries are usually poor and coarse, which aggravates the touching problem \cite{Dou2016,Chen2017,Ronneberger2015,Lieman-Sifry2017,Chen2016c}.
To this end, the U-net was proposed by employing a U-shape deep convolutional network for biomedical image segmentation and obtained state-of-the-art performance on several challenges.
Although the raw context information can be directly propagated to higher resolution layers for detail preserving by U-net, the utilization of weighting losses on boundaries may not satisfactorily handle the touching problem.
Soon afterwards, several improvement on U-net were proposed \cite{Lieman-Sifry2017,Chen2016c,Cicek2016}.
\cite{Chen2016c} proposes kU-net by employing multiple submodule U-nets to work on different image scales. 
\cite{Cicek2016} extends the U-net to 3D applications and \cite{Lieman-Sifry2017} substitutes the same padding with valid padding in cardiac segmentation.
\cxj{a different saying with that in introduction.}
%
Recently, DCAN~\cite{Chen2017} is proposed as a multi-task learning framework to solve the touching problem by incorporating complementary object contours into the model, and obtained state-of-the-art performance in the challenging gland segmentation task.
\mdf{Inspired by DCAN, our DSAN ....} \cxj{discribe our main idea.}

\noindent \textbf{Shape constraints for segmentation} Instead of directly learning from raw images, many methods have been attempted to introduce the prior knowledge of plausible objects.
\cite{Delgado-Gonzalo2012} was designed for segmenting elliptical objects sequentially and \cite{Veksler2008a} introduced the star convexity priors into their model by restricting all rays emanating from a user-defined central point.
Soon, the approach in \cite{Veksler2008a} was extended to Geodesic Star Convexity constraint by \cite{Gulshan2010a}, and \cite{Gorelick2014a} succeeded in introducing the convex shape constraints into segmentation task.
Recently, \cite{Royer2016a} is able to handle many generic convex objects of multiple foreground classes automatically.
Although, the existing methods gave the satisfying performance in their task, they did use the powerful feature representation capability of CNN model. 
\cxj{they did use or did NOT use?}
Thus in this paper, we \cxj{first?} introduce the shape constraint into the neural network for biomedical segmentation, of which the prior shape knowledge about plausible objects are usually available.
